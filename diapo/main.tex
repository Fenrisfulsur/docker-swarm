\documentclass{cubeamer}

\usepackage[french]{babel}
\usepackage[T1]{fontenc}
\usepackage{multicol}
\usepackage{hyperref}
\usepackage{url}
\usepackage{xcolor}
% Code enlightement
\usepackage{code-package}

\title{Clustering avec Docker Swarm}
\subtitle{Administration et protocoles réseaux}
\author[Sami Babigeon \& Louka Boivin]{Sami Babigeon \& Louka Boivin}
\date{\today}
\institute[Université de Rouen]{Master 2 SSI}

% Turn off slides numbering with allowframebreaks
\setbeamertemplate{frametitle continuation}{}
% Right arrow
\newcommand{\arrow}{$\rightarrow$ }

\begin{document}

% Title and ToC

\maketitle

\begin{frame}[fragile]{Mise en place}
Télécharger le script d'installation sur:
\begin{bashWithOption}{minted options = {fontsize=\footnotesize}}
wget https://raw.githubusercontent.com/Fenrisfulsur/docker-swarm/main/setup.sh     
chmod u+x setup.sh
sudo ./setup.sh install
\end{bashWithOption}
\end{frame}

\cutoc

% Main section
\section{Docker}

\begin{frame}{Présentation}
    \begin{multicols}{2}
    Docker est une plateforme open-source permettant de concevoir, tester et déployer rapidement
    des applications.
    
    Il utilise la notion de conteneurs qui, comme la virtualisation,
    permet d'isoler une application du serveur hôte tout en étant plus léger et plus simple à
    déployer.

    \columnbreak
    \begin{figure}
        \centering
        \includegraphics[width=6cm]{img/docker}
    \end{figure}    

    \end{multicols}
\end{frame}

\begin{frame}{Fonctionnement}
    \begin{multicols}{2}
    Docker étend les conteneurs LXC qui s'appuient sur les fonctionnalités du noyau pour
    l'isolation (namespaces, cgroups), en fournissant une API haut niveau, plus simple
    d'utilisation.

    Le Docker Hub est une bibliothèque d'images publics pour créer facilement tous
    types de conteneurs.
    
    \columnbreak
    \begin{figure}
        \centering
        \includegraphics[width=7cm]{img/vm-container}
    \end{figure}    

    \end{multicols}
\end{frame}

\begin{frame}[fragile]{Installation}
    \begin{multicols}{2}
Sur une machine Ubuntu :
\begin{bashResized}{0.48}
sudo apt install docker-ce\
 docker-ce-cli containerd.io
\end{bashResized}
    \columnbreak
    
Sur MacOS ou Windows :
\begin{bashResized}{0.48}
Installer Docker Desktop :
https://www.docker.com/
\end{bashResized}
    \end{multicols}

    Pour pouvoir lancer des commandes docker sans privilèges :
\begin{bash}
sudo usermod -aG docker <username>
\end{bash}
\end{frame}

\begin{frame}[fragile]{Utilisation}
    Puis ensuite pour tester :
\begin{bash}
docker pull python
docker run -it python
\end{bash}

    \begin{figure}
        \centering
        \includegraphics[width=\textwidth]{img/docker-test.png}
    \end{figure}
\end{frame}

\section{Docker Swarm}

% Multicol frame
\begin{frame}{Présentation}
    \begin{multicols}{2}
    Docker Swarm est un outil d'orchestration permettant de créer et gérer des clusters de
    machines physiques ou virtuelles hébergeant une application. C'est un outil intégré à Docker.

    L'interêt principal d'un cluster Swarm est la haute disponibilité que cela procure pour
    l'application.

    \columnbreak
    \begin{figure}
        \centering
        \includegraphics[width=4cm]{img/swarm}
    \end{figure}

    \end{multicols}
\end{frame}

\begin{frame}{Terminologie}
    \begin{multicols}{2}
        \begin{itemize}
            \item Éléments d'un cluster \arrow \emph{Noeuds (nodes)}
            \item Cluster \arrow \emph{Manager(s) + Workers}
            \item Workers \arrow \emph{Exécution}
            \item Managers \arrow \emph{Orchestration}
        \end{itemize}
    \columnbreak
        \begin{figure}
            \centering
            \includegraphics[width=0.5\textwidth]{img/manager}
        \end{figure}
    \end{multicols}
\end{frame}

\begin{frame}{Schéma d'un cluster}
    \begin{figure}
        \centering
        \includegraphics[width=0.85\textwidth]{img/swarm-network}
    \end{figure}
\end{frame}

\begin{frame}{Pourquoi utiliser Swarm ?}
    \begin{multicols}{2}
        \begin{itemize}
            \item Évolutivité
            \item Équilibrage de charges automatique
            \item Retour à l'état souhaité
            \item Mise en réseau multi-hôtes
            \item Sécurité
        \end{itemize}
    \columnbreak
        \begin{figure}
            \centering
            \includegraphics[width=0.24\textwidth]{img/scalability}
            \includegraphics[width=0.24\textwidth]{img/load-balancing}
        \end{figure}
    \end{multicols}
\end{frame}

\begin{frame}{Sécurité}
TLS - PKI - CA externe

Logs du Raft Consensus chiffrés

Utilisation de la fonctionnalité d'\verb:autolock:
\end{frame}

\section{Démonstration}

\begin{frame}[fragile]{Création du cluster}
Se connecter sur \verb:docker-manager: puis lancer:
\begin{bash}
docker swarm init
\end{bash}
\end{frame}

\begin{frame}[fragile]{Ajout des nodes}
Se connecter sur \verb:docker-worker1: et \verb:docker-worker2: puis lancer:
\begin{bash}
docker swarm join --token <token> <manager ip>:2377
\end{bash}

On peut voir l'état du cluster avec :
\begin{bash}
docker node ls
\end{bash}
\end{frame}

\begin{frame}[fragile]{Création du service web}
Se connecter sur \verb:docker-manager: puis lancer:
\begin{bash}
docker service create --name my_web --replicas 3 \
    --mount type=bind,source=/,target=/usr/share/nginx/html \
    --publish 8080:80 nginx
\end{bash}
\end{frame}

\begin{frame}[fragile]{Tests}
Depuis l'hôte:
\begin{bash}
for i in `seq 1 10`; do curl http://<any ip>:8080/index.html; done
\end{bash}
\end{frame}

\begin{frame}{Équilibrage de charge}
    \begin{figure}
        \centering
        \includegraphics[width=\textwidth]{img/routing-mesh}
    \end{figure}
\end{frame}

\begin{frame}[fragile]{Disponibilité}
Depuis l'hôte :
\begin{bash}
sudo lxc-stop docker-worker2
\end{bash}

Ensuite, depuis \verb:docker-manager: :
\begin{bash}
docker service ps my_web
\end{bash}
\end{frame}

\begin{frame}[fragile]{Sécurité - \texttt{autolock}}
Depuis le \verb:docker-manager: :
\begin{bash}
docker swarm update --autolock=true
\end{bash}

On redémarre Docker et on inspecte le service:
\begin{bash}
sudo systemctl restart docker; docker service ls
\end{bash}

On peut ensuite déverrouiller le cluster avec :
\begin{bash}
docker swarm unlock
\end{bash}
\end{frame}

\section{Conclusion}

% Q&A
\begin{frame}[standout]
    \Huge\textsc{Merci de votre écoute}
    \vfill
    \LARGE\textsc{Questions ?}
\end{frame}

\section*{Bibliographie}

\begin{frame}[allowframebreaks]
    \frametitle{Références}
    \nocite{*}
    \bibliographystyle{acm}
    \bibliography{bibliography}
\end{frame}

\end{document}

